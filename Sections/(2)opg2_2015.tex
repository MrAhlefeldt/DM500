\documentclass{article}
\usepackage{amsmath, amssymb}

\begin{document}

%\section{Reeksamen 2015 - Februar}

\begin{center}
    \huge\bfseries Reeksamen 2015 - Februar \\
\end{center}

Opgave 2 (18\%)

Hvilke af følgende udsagn er sande?
\[
\forall x\in\mathbb{N}:\exists y\in\mathbb{N}:x<y
\]
Sandt, da for alle $x$ i de naturlige tal, så kan der altid findes en $y$-værdi der er større end $x$.

\[
\forall x\in\mathbb{N}: \exists! y\in\mathbb{N}:x<y
\]
Falsk, da der eksisterer flere $y$-værdier der er større end $x$. Et andet argument for at den er falsk er fordi 1) er sandt, så vil 2) være falsk, da de modsiger hinanden med mængden af $y$-værdier.

\[
\exists y\in\mathbb{N}:\forall x\in\mathbb{N}:x<y
\]
Udsagnet siger at der eksisterer et naturligt tal $y$ hvor for alle naturlige tal $x$, $x$ er mindre end $y$, så udsagnet er sandt.
\newline

Angiv negationen af udsagn 1. fra spørgsmål a).
Negerings-operatoren må ikke indgå i dit udsagn.
\newline

Vores udsagn fra 1) siger: ”For alle naturlige tal $x$, eksisterer der et naturligt tal $y$ sådan at $x$ er mindre end $y$.”, men vores negation af udsagnet ville så sige: ”Der eksisterer et naturligt tal $x$ sådan at for hvert naturligt tal $y$, $x$ er større eller lig med $y$.”

\[
\exists x\in\mathbb{N}:\forall y\in\mathbb{N}: x\geq y
\]

\end{document}