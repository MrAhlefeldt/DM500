\begin{center}
    \huge\bfseries Reeksamen 2015 - Februar \\
\end{center}

Opgave 1 (12\%) \\
I det følgende lader vi $U=\{1,2,3,\ldots,15\}$ være universet (universal set).
Betragt de to mængder:
\[ A=\{2n \,|\, n\in S\} \text{ og } B=\{3n+2 \,|\, n\in S\} \]
Hvor 
\[S=\{1,2,3,4\}\]
Angiv samtlige elementer i hver af følgende mængder:

$a)$: $A$

\hspace{15pt}   Vi erstatter værdien $n$ fra $A$ med $S$:
\[ A=\{1\cdot2,2\cdot2,3\cdot2,4\cdot2\} \]
\[ A=\{2,4,6,8\} \]

 $b)$: $B$
 
   \hspace{15pt} Vi erstatter værdien $n$ fra $B$ med $S$:
\[ B=\{3\cdot1+2,3\cdot2+2,3\cdot3+2,3\cdot4+2\} \]
\[ B=\{5,8,11,14\} \]

$c)$: $A\cap B$

    $A\cap B$  er fællesmængden af  $A=\{2,4,6,8\}$ og $B=\{5,8,11,14\}$. Både $A$ og $B$ indeholder elementet, så fællesmængden for  $A\cap B$ er:
    \[A\cap B=\{8\}\]

$d)$: $A\cup B$

\hspace{15pt}    $A\cup B$   kan beskrives som at være en kombination af alle  de mængder både $A$ og $B$ har uden dubletter. Herved kan vi sige at:
\[ A\cup B=\{2,4,6,8,11,14\} \]

$e)$: $A-B$ 
    
    Vi fjerner alle elementer fra  $A$ som deles med $B$: 
\[ A-B=\{2,4,6\} \]
$f)$:  $\overline{A}$ 

    Vores universalmængde er $U=\{1,2,3,\ldots,15\}$ og $A=\{2,4,6,8\}$.
    Den komplementære mængde af $A$ vil så være:
\[ \overline{A}=\{1,3,5,7,9,10,11,12,13,14,15\} \]